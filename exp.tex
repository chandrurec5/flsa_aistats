\section{Numerical Experiments}\label{sec:exp}
\FloatBarrier
\begin{figure*}
\caption{The plot in the left shows that rates achievable for the additive noise case. The plot in the middle shows performance of TD(0) and GTD using normalized $5^{th}$ order Fourier basis in the mountain car domain. The right most plot shows the performance of GTD algorithm for two different  step-size choices in the Baird domain.}
\label{fig:results}

\begin{tabular}{ccc}
\begin{minipage}{0.6\columnwidth}
 \resizebox{1\columnwidth}{!}{
\begin{tikzpicture} 
\begin{loglogaxis}
    [
      error bars/y dir      = both,
      error bars/y explicit = true,
      legend pos=north west
    ]
        \addplot[only marks, mark=+] table[y ]  {est/est};
    \addplot[only marks, mark=o] table[y ]  {pred/pred};
        \addplot[only marks, mark=*] table[y ]  {struc/struc};
\addlegendentry{$ee$}
\addlegendentry{$peusc$}
\addlegendentry{$pesc$}
%\addplot[smooth] table[y error index = 2]  {tdon/tile1};    

\end{loglogaxis}
 \end{tikzpicture}
  }
 \end{minipage}
&
\begin{minipage}{0.6\columnwidth}
 \resizebox{1\columnwidth}{!}{
\begin{tikzpicture} 
\begin{axis}
    [
      error bars/y dir      = both,
      error bars/y explicit = true,
      legend pos=north west
    ]
        \addplot[only marks, mark=+] table[y error index=2]  {tdon/fouriern1};
    \addplot[only marks, mark=o] table[y error index =2]  {gtdon/fouriern1};
\addlegendentry{TD(0)}
\addlegendentry{GTD}
%\addplot[smooth] table[y error index = 2]  {tdon/tile1};    

\end{axis}
 \end{tikzpicture}
  }
 \end{minipage}

&
\begin{minipage}{0.6\columnwidth}
 \resizebox{1\columnwidth}{!}{
\begin{tikzpicture} 
\begin{axis}
    [
      error bars/y dir      = both,
      error bars/y explicit = true,
      legend pos=north west
    ]
        \addplot[only marks, mark=+] table[y ]  {baird/gtd};
    \addplot[only marks, mark=o] table[y ]  {baird/gtd_cs};
\addlegendentry{$\alpha=0.005,\beta=16$}
\addlegendentry{$\alpha=\frac{1}{14},\beta=1$}
%\addplot[smooth] table[y error index = 2]  {tdon/tile1};    

\end{axis}
 \end{tikzpicture}
  }
 \end{minipage}



\end{tabular}
\end{figure*}



\begin{comment}
 \begin{minipage}{0.6\columnwidth}

\resizebox{1\columnwidth}{!}{
\begin{tabular}{|c||l|l|l||l|}
  \hline
  \multirow{2}{*}{\emph{on-policy}} 
      & \multicolumn{2}{c|}{TD(0)} 
          & \multicolumn{2}{c|}{GTD} \\             \cline{2-5}
  & $2e3$ & $2e4$ & $2e3$ & $2e4$ \\  \hline
  $Tile~1$ & $1.35e^{-1}$ & $1.23e^{-1}$ & NA & NA \\      \hline
  $Tile~2$ & $1.27e^{-1}$ & $1.10e^{-1}$ & NA & NA \\      \hline
  $Tile~3$ & $1.73e^{-1}$ & $1.13e^{-1}$ & NA & NA \\      \hline
  $Tile~4$ & $1.47e^{-1}$ & $1.07e^{-1}$ & NA & NA \\      \hline
  $FUN~1$ & $1.45e^{-1}$ & $1.10e^{-1}$ & $7.19e^{-1}$& $2.12e^{-1}$ \\      \hline
  $FUN~2$ & $1.77e^{-1}$ & $1.05e^{-1}$ & $7.75e^{-1}$ & $2.24e^{-1}$ \\      \hline
  $FUN~3$ & $2.42e^{-1}$ & $1.09e^{-1}$ & $8.06e^{-1}$ & $2.82e^{-1}$ \\      \hline
  $FUN~4$ & $3.35e^{-1}$ & $1.16e^{-1}$ & $8.50e^{-1}$ & $3.39e^{-1}$ \\      \hline
  $FN~1$ & $1.22e^{-1}$ & $1.09e^{-1}$ & $8.25e^{-1}$ & $3.40e^{-1}$ \\      \hline
  $FN~2$ & $1.24e^{-1}$ & $1.03e^{-1}$ & $8.90e^{-1}$ & $4.33e^{-1}$ \\      \hline
  $FN~3$ & $1.62e^{-1}$ & $1.08e^{-1}$ & $9.21e^{-1}$ & $5.43e^{-1}$ \\      \hline
  $FN~4$ & $2.22e^{-1}$ & $1.09e^{-1}$ & $9.38e^{-1}$ & $6.25e^{-1}$ \\      \hline
   
\end{tabular}
}

  
 \end{minipage}
&
 \begin{minipage}{0.6\columnwidth}
 

 \resizebox{\columnwidth}{!}{
\begin{tabular}{|c||l|l|l||l|}
  \hline
  \multirow{2}{*}{\emph{off-policy}} 
      & \multicolumn{2}{c|}{TD(0)} 
          & \multicolumn{2}{c|}{GTD} \\             \cline{2-5}
  & $5e4$ & $5e5$ & $5e4$ & $5e5$ \\  \hline
  $Tile~1$ & $6.48e^{-2}$ & $6.52e^{-2}$ & NA & NA \\      \hline
  $Tile~2$ & $4.25-2$ & $2.81e^{-2}$ & NA & NA \\      \hline
  $Tile~3$ & $8.50e^{-2}$ & $3.38e^{-2}$ & NA & NA \\      \hline
  $Tile~4$ & $1.04e^{-1}$ & $2.70e^{-2}$ & NA & NA \\      \hline
  $FUN~1$ & $9.59e^{-2}$ & $6.40e^{-2}$ & $7.73e^{-1}$& $3.22e^{-1}$ \\      \hline
  $FUN~2$ & $1.53e^{-1}$ & $1.91e^{-1}$ & $8.41e^{-1}$ & $4.87e^{-1}$ \\      \hline
  $FUN~3$ & $1.77e^{-1}$ & $1.98e^{-1}$ & $8.62e^{-1}$ & $5.41e^{-1}$ \\      \hline
  $FUN~4$ & $3.04e^{-1}$ & $1.48e^{-1}$ & $8.87e^{-1}$ & $6.21e^{-1}$ \\      \hline
  $FN~1$ & $1.49e^{-1}$ & $1.63e^{-2}$ & $9.33e^{-1}$ & $5.75e^{-1}$ \\      \hline
  $FN~2$ & $1.46e^{-1}$ & $1.91e^{-1}$ & $9.71e^{-1}$ & $7.82e^{-1}$ \\      \hline
  $FN~3$ & $1.16e^{-1}$ & $2.01e^{-1}$ & $9.78e^{-1}$ & $8.68e^{-1}$ \\      \hline
    $FN~4$ & $1.40e^{-1}$ & $1.50e^{-1}$ & $9.88e^{-1}$ & $9.08e^{-1}$ \\      \hline
    
\end{tabular}
}

 
 \end{minipage}
\end{comment}
\paragraph{Finite-Time Rates:} We first present results that support the arguments made in \Cref{sec:land}. To this end, we consider two problem classes $\P_{usc}$, $\P_{sc}$ described as follows: Let $d=2$, for any $P\in \P_{usc}$, $A_P=\left[\begin{matrix}1 &0\\ 0 & a_p\end{matrix}\right]$, for some $a_p>0$, let $b_P=\left[\begin{matrix}0\\0\end{matrix}\right]$, $A_t=A_P,\forall t\geq 0$and $N_t=\left[\begin{matrix}\zeta(1)_t\\ \zeta(2)_t\end{matrix}\right]$, where $\zeta(i)_t, i=1,2$ are \iid uniform random variable taking values in $[-1,1]$. $\P_{sc}$ is the same as $\P_{usc}$ except that, in $\P_{sc}$ the noise $N_t=A_P \left[\begin{matrix}\zeta(1)_t\\ \zeta(2)_t\end{matrix}\right]$, i.e., the noise scales with $A_P$.

Notice that the classes $\P_{usc}$ and $\P_{sc}$ have only additive noise, and are symmetric positive definite. Further,  both $\P_{usc}$ and $\P_{sc}$ are admissible with any $\alpha\in(0,1)$ as a witness. The \Cref{fig:results} (left most plot) shows $ee=\E{\norm{\thh_t-\ts}^2}$ (estimation error) and $peusc=\E{\norm{\thh_t-\ts}^2}_{A_P}$ (prediction error without scaled noise) measured for the class $\P_{usc}$,  and $pesc=\E{\norm{\thh_t-\ts}^2}_{A_P}$ ( prediction error with scaled noise)  measured for the class $\P_{sc}$. for $t=\{10,20,50,100,200,1000\}$ and \emph{adversarial} choices of $A_P$ form the class, i.e., each point in any of the three curves ($ee/peusc/pesc$) corresponds to an appropriate adversarial choice of $A_P$. 

In all the plots, $ee/peusc/pesc$ in addition to adversarial choice of $A_P$, the condition $\Lambda(\alpha A_P t )<1$ as in \Cref{prop:erradd} was also ensured. 
To obtain $ee$, we let $\alpha=0.9$, $a_P=\frac{1}{t}$. We observe that $ee$ increases $O(t)$. To obtain $peusc$, the adversarial choice $A_P=\frac{1}{\sqrt{t}}$ was chose for a given $t$ and $\alpha=\frac{1}{t}$ was chosen. We observe that $peusc$ decreases only at $O(\frac{1}{\sqrt{t}})$ as remarked in \Cref{sec:land}. To obtain $pesc$, we chose $\alpha=0.9$ and $A_P=\frac{1}{t}$, and it obeys the $O(\frac{1}{t})$ rate.

\paragraph{Mountain Car}   
The mountain car is a widely used domain for policy improvement tasks, however, here, we use the domain for approximate policy evaluation. The domain consists of an under-powerd car, that needs to swing from the bottom of the valley to the top by performing either one of the three possible actions: \emph{forward, reverse,no} throttle. Since the car is under-powered, it cannot directly accelerate to the top from the bottom and needs to swing back and forth to reach the top. The state of the system is described the position and the velocity of the car at a given time. For the purpose of \emph{on-policy} evaluation, we sample from the policy $\pi$ that accelerates in the direction of the velocity with probability $\frac{298}{300}$ and the other two actions with probability $\frac{2}{300}$. Since, we are also interested in the \emph{off-policy} case sample using a behaviour policy  $\pi_b$ that accelerate in the direction of the velocity with probability $\frac{8}{10}$ and chooses the other two actions with probability $\frac{2}{10}$. Thus the importance sampling ratio can be as high as ($\rho_{\max}$) $30$. 
 We used \emph{Tile coding} and \emph{Fourier} basis (un-normalized and normalized). We used $4$ different tiling ($4\times 4$ and $7\times 7$ gird for the two state variables permuted with $5$ and $10$ tiles), and we also tried $4$ different $n^{th}$ fourier basis function ($n=3,5,7,9$), with $d=(n+1)^2$. For a given state $s=(p,v)$\footnote{We scale the states by subtracting the minimum value and dividing it by its range, so that $p,v\in(0,1)$ after scaling}, the Fourier feature is given by $\phi(p,v)=\big(cos(\pi [c_1p+c_2 v],c_1,c_2=0,1,\ldots,n)\big) \in \R^{d}$ (where $d=(n+1)^2$). The normalized features were obtained by letting $\norm{\phi(s)}^2_2=1$. We generated $100$ trajectories for the \emph{on/off}-policies, and the discount factor we used was $\gamma=0.999$. Further, we also calculated the true value function at a fixed (but randomly chosen) set of $40$ states from the trajectories by separately simulating trajectories starting from those states and computing the discounted sum (for each such state we averaged over $100$ runs).
 We would like to mention some important observations. The step-size choices for the various experiments can be found in the \Cref{tab:step-size}, and follows directly from \Cref{th:tdadmis}. Firstly, in the cases TD(0)/GTD \emph{on/off}-policy there was no divergence or instability, which is reassuring given the fact that across the $48$ settings there was no instability issue by choosing step-size recommended by \Cref{th:tdadmis}. The squared errors can be found in the tables in \Cref{fig:results}. We compared the squared error between the approximate value function and the correct value function at $40$ randomly chosen points from the trajectory, and normalized the values with respect to the norm of the true value function (i.e., the norm of the $40$-dimensional vector). 
 
 \paragraph{Slowness of GTD:}  GTD was slower with tile coding and unnormalized Fourier basis in the \emph{on/off}-policy settings. This might be due to the relatively smaller step-size for GTD in comparison to TD(0). However, with normalized features it was comparable to TD(0) (however GTD was slower even in this case). The speed issue can be related to the fact that spectrum of GTD involves eigenvalues of the $A^\top A$  and hence any small eigenvalue of $A$ matrix gets squared results in a degradation of speed in comparison to TD(0).

\begin{table}
\resizebox{\columnwidth}{!}{
\begin{tabular}{|c|c|c|c|}\hline
\backslashbox{Policy}{Basis} & Tile& \makecell{Fourier\\ Un-normalized} &\makecell{Fourier \\ Normalized}\\ \hline
	TD(0)\emph{on}& $\frac1k$	&$\frac1d$ & $1$\\ \hline
	TD(0)\emph{off}& $\frac1{\rho_{\max}k}$	& $\frac1{\rho_{\max}d}$ & $\frac1{\rho_{\max}}$\\\hline			
	GTD\emph{on}& $\frac1{2k^2}$	&$\frac1{2d^2}$ & $\frac{1}{2}$\\ \hline
	GTD\emph{off}& $\frac1{\rho_{\max}k^2}$	& $\frac1{\rho_{\max}d^2}$ & $\frac1{\rho_{\max}2}$\\\hline			

\end{tabular}
}
\label{tab:step-size}
\caption{step-size choices for the \emph{on/off}-policy settings in the mountain car experiment. The same step-size rule was followed for TD as well as GTD. Notice that these step-size choices directly follow from the results in \Cref{th:tdadmis} without any further tuning.}
\end{table}
%\FloatBarrier
\begin{figure*}[t]
\resizebox{\textwidth}{!}{
\begin{tabular}{ccc}
\begin{tikzpicture} 
\begin{axis}
    [
      error bars/y dir      = both,
      error bars/y explicit = true,
    ]
    \addplot[smooth] table[y error index = 2]  {tdon/fouriern1};
        \addplot[smooth] table[y error index = 2]  {tdon/fouriern2};
            \addplot[smooth] table[y error index = 2]  {tdon/fouriern3};
                \addplot[smooth] table[y error index = 2]  {tdon/fouriern4};
\end{axis} \end{tikzpicture}
&

\begin{tikzpicture} 
\begin{axis}
    [
      error bars/y dir      = both,
      error bars/y explicit = true,
    ]
    \addplot[smooth] table[y error index = 2]  {tdon/fourierun1};
        \addplot[smooth] table[y error index = 2]  {tdon/fourierun2};
            \addplot[smooth] table[y error index = 2]  {tdon/fourierun3};
                \addplot[smooth] table[y error index = 2]  {tdon/fourierun4};
\end{axis} \end{tikzpicture}
&

\begin{tikzpicture} 
\begin{axis}
    [
      error bars/y dir      = both,
      error bars/y explicit = true,
    ]
    \addplot[smooth] table[y error index = 2]  {tdon/tile1};
        \addplot[smooth] table[y error index = 2]  {tdon/tile2};
            \addplot[smooth] table[y error index = 2]  {tdon/tile3};
                \addplot[smooth] table[y error index = 2]  {tdon/tile4};
\end{axis} \end{tikzpicture}

\end{tabular}
}
\end{figure*}





\FloatBarrier
\begin{figure*}[t]
\resizebox{\textwidth}{!}{
\begin{tabular}{ccc}
\begin{tikzpicture} 
\begin{axis}
    [
      error bars/y dir      = both,
      error bars/y explicit = true,
    ]
    \addplot[smooth] table[y error index = 2]  {tdoff/fouriern1};
        \addplot[smooth] table[y error index = 2]  {tdoff/fouriern2};
            \addplot[smooth] table[y error index = 2]  {tdoff/fouriern3};
                \addplot[smooth] table[y error index = 2]  {tdoff/fouriern4};
\end{axis} \end{tikzpicture}
&

\begin{tikzpicture} 
\begin{axis}
    [
      error bars/y dir      = both,
      error bars/y explicit = true,
    ]
    \addplot[smooth] table[y error index = 2]  {tdoff/fourierun1};
        \addplot[smooth] table[y error index = 2]  {tdoff/fourierun2};
            \addplot[smooth] table[y error index = 2]  {tdoff/fourierun3};
                \addplot[smooth] table[y error index = 2]  {tdoff/fourierun4};
\end{axis} \end{tikzpicture}
&

\begin{tikzpicture} 
\begin{axis}
    [
      error bars/y dir      = both,
      error bars/y explicit = true,
    ]
    \addplot[smooth] table[y error index = 2]  {tdoff/tile1};
        \addplot[smooth] table[y error index = 2]  {tdoff/tile2};
            \addplot[smooth] table[y error index = 2]  {tdoff/tile3};
                \addplot[smooth] table[y error index = 2]  {tdoff/tile4};
\end{axis} \end{tikzpicture}

\end{tabular}
}
\end{figure*}


\FloatBarrier
\begin{figure*}[t]
\resizebox{\textwidth}{!}{
\begin{tabular}{ccc}
\begin{tikzpicture} 
\begin{axis}
    [
      error bars/y dir      = both,
      error bars/y explicit = true,
    ]
    \addplot[smooth] table[y error index = 2]  {gtdon/fouriern1};
        \addplot[smooth] table[y error index = 2]  {gtdon/fouriern2};
            \addplot[smooth] table[y error index = 2]  {gtdon/fouriern3};
                \addplot[smooth] table[y error index = 2]  {gtdon/fouriern4};
\end{axis} \end{tikzpicture}
&

\begin{tikzpicture} 
\begin{axis}
    [
      error bars/y dir      = both,
      error bars/y explicit = true,
    ]
    \addplot[smooth] table[y error index = 2]  {gtdon/fourierun1};
        \addplot[smooth] table[y error index = 2]  {gtdon/fourierun2};
            \addplot[smooth] table[y error index = 2]  {gtdon/fourierun3};
                \addplot[smooth] table[y error index = 2]  {gtdon/fourierun4};
\end{axis} \end{tikzpicture}
&

\begin{tikzpicture} 
\begin{axis}
    [
      error bars/y dir      = both,
      error bars/y explicit = true,
    ]
    \addplot[smooth] table[y error index = 2]  {gtdon/tile1};
        \addplot[smooth] table[y error index = 2]  {gtdon/tile2};
            \addplot[smooth] table[y error index = 2]  {gtdon/tile3};
                \addplot[smooth] table[y error index = 2]  {gtdon/tile4};
\end{axis} \end{tikzpicture}

\end{tabular}
}
\end{figure*}

\FloatBarrier
\begin{figure*}[t]
\resizebox{\textwidth}{!}{
\begin{tabular}{ccc}
\begin{tikzpicture} 
\begin{axis}
    [
      error bars/y dir      = both,
      error bars/y explicit = true,
    ]
    \addplot[smooth] table[y error index = 2]  {gtdoff/fouriern1};
        \addplot[smooth] table[y error index = 2]  {gtdoff/fouriern2};
            \addplot[smooth] table[y error index = 2]  {gtdoff/fouriern3};
                \addplot[smooth] table[y error index = 2]  {gtdoff/fouriern4};
\end{axis} \end{tikzpicture}
&

\begin{tikzpicture} 
\begin{axis}
    [
      error bars/y dir      = both,
      error bars/y explicit = true,
    ]
    \addplot[smooth] table[y error index = 2]  {gtdoff/fourierun1};
        \addplot[smooth] table[y error index = 2]  {gtdoff/fourierun2};
            \addplot[smooth] table[y error index = 2]  {gtdoff/fourierun3};
                \addplot[smooth] table[y error index = 2]  {gtdoff/fourierun4};
\end{axis} \end{tikzpicture}
&

\begin{tikzpicture} 
\begin{axis}
    [
      error bars/y dir      = both,
      error bars/y explicit = true,
    ]
    \addplot[smooth] table[y error index = 2]  {gtdoff/tile1};
        \addplot[smooth] table[y error index = 2]  {gtdoff/tile2};
            \addplot[smooth] table[y error index = 2]  {gtdoff/tile3};
                \addplot[smooth] table[y error index = 2]  {gtdoff/tile4};
\end{axis} \end{tikzpicture}

\end{tabular}
}
\end{figure*}

\paragraph{Baird} In this domain there are $S=\{s_1,\ldots,s_7\}$ states and $A=\{a_1,a_2\}$ actions. Under, $a_1$ we have $p_{a_1}(s,s_1)=1, \forall s\in S$ and under $a_2$ we have $p_{a_2}(s,s')=\frac{1}{6}, \forall s\in S, s'=2,\ldots,7$. The samples are collected using a behaviour policy $\pi_b$ that performs action $a_2$ with probability $\frac{6}{7}$ and action $a_1$ with probability $\frac17$, and the target policy that we are interested is $\pi$ which performs action $a_1$ in all the states. This example is a case where $\E{\phi_t\phi_t}\neq\E{\phi'_t{\phi'_t}^\top}$ and hence the result for TD(0) in \Cref{th:tdadmis} does not hold. The feature vector we chose was: $\phi(s_1)=[\frac{1}{2}\,0\,0\,0\,0\,0\,0\,\mu]$, $\phi(s_i)=e_i+[0\,0\,0\,0\,0\,0\,0\,\frac{1}{2}], i=2,\ldots,7$, where $e_i$ is the standard basis with $i^{th}$ co-ordinate $1$ and rest of the co-ordinates $0$. It is well known that for $\mu=1$, TD(0) diverges in the \emph{off-}policy setting in this domain. We have the following interesting observations in this domain:
\begin{enumerate}[leftmargin=*]
\item For $\mu=\frac{1}{2}$ we observed that $TD(0)$ does not diverge even when $\E{\phi_t\phi_t}=\E{\phi'_t{\phi'_t}^\top}$ is not met (this means the condition is not \emph{necessary} for \emph{off-}policy divergence). However the error did not converge to $0$ because we found that for this of $\beta$, the underlying $A$ matrix has an eigenvalue very close to $0^+$ and the corresponding sub-space remains as residual error.
\item  \citet{gtdmp} mention that\footnote{We believe that this choice in fact comes from \cite{dann} (see Figure $23$d) } for the BAIRD domain $\alpha=0.005$ (and $\beta=16$ which is the ratio of the step-sizes between primal and dual variable)is the optimal step-size. We compared the performance of GTD with $\alpha=0.005$ (and $\beta=16$) with the choice of $\alpha=\frac{1}{7\times 2}$ ($7$ is $\rho_{\max}$ and $2$ is because the chosen feature has only $2$ co-ordinates non-zero for any given state) and initial condition $\theta_0=[1\, 1\, 1\, 1\, 1\, 1\, 10\, 1]$. Further, the same observation continued to hold for random initialization of features.
\end{enumerate}
%is an example domain where $\E{\phi_t\phi_t^\top}\neq \E{\phi_t{\phi'}_t^\top}$, and hence the result for TD(0) in \Cref{th:tdadmis} does not hold. 


\begin{comment}
\begin{table}
\begin{tabular}{|c||l|l|l||l|}
  \hline
  \multirow{2}{*}{\emph{on-policy}} 
      & \multicolumn{2}{c|}{TD(0)} 
          & \multicolumn{2}{c|}{GTD} \\             \cline{2-5}
  & $2e3$ & $2e4$ & $2e3$ & $2e4$ \\  \hline
  $Tile~1$ & $1.35e-1$ & $1.23e-1$ & NA & NA \\      \hline
  $Tile~2$ & $1.27e-1$ & $1.10e-1$ & NA & NA \\      \hline
  $Tile~3$ & $1.73e-1$ & $1.13e-1$ & NA & NA \\      \hline
  $Tile~4$ & $1.47e-1$ & $1.07e-1$ & NA & NA \\      \hline
  $FUN~1$ & $1.45e-1$ & $1.10e-1$ & $7.19e-1$& $2.12e-1$ \\      \hline
  $FUN~2$ & $1.77e-1$ & $1.05e-1$ & $7.75e-1$ & $2.24e-1$ \\      \hline
  $FUN~3$ & $2.42e-1$ & $1.09e-1$ & $8.06e-1$ & $2.82e-1$ \\      \hline
  $FUN~4$ & $3.35e-1$ & $1.16e-1$ & $8.50e-1$ & $3.39e-1$ \\      \hline
  $FN~1$ & $1.22e-1$ & $1.09e-1$ & $8.25e-1$ & $3.40e-1$ \\      \hline
  $FN~2$ & $1.24e-1$ & $1.03e-1$ & $8.90e-1$ & $4.33e-1$ \\      \hline
  $FN~3$ & $1.62e-1$ & $1.08e-1$ & $9.21e-1$ & $5.43e-1$ \\      \hline
  $FN~4$ & $2.22e-1$ & $1.09e-1$ & $9.38e-1$ & $6.25e-1$ \\      \hline
    
\end{tabular}
\end{table}





\begin{table}
\begin{tabular}{|c||l|l|l||l|}
  \hline
  \multirow{2}{*}{\emph{on-policy}} 
      & \multicolumn{2}{c|}{TD(0)} 
          & \multicolumn{2}{c|}{GTD} \\             \cline{2-5}
  & $5e4$ & $5e5$ & $5e4$ & $5e5$ \\  \hline
  $Tile~1$ & $6.48e-2$ & $6.52e-2$ & NA & NA \\      \hline
  $Tile~2$ & $4.25-2$ & $2.81e-2$ & NA & NA \\      \hline
  $Tile~3$ & $8.50e-2$ & $3.38e-2$ & NA & NA \\      \hline
  $Tile~4$ & $1.04e-1$ & $2.70e-2$ & NA & NA \\      \hline
  $FUN~1$ & $9.59e-2$ & $6.40e-2$ & $7.73e-1$& $3.22e-1$ \\      \hline
  $FUN~2$ & $1.53e-1$ & $1.91e-1$ & $8.41e-1$ & $4.87e-1$ \\      \hline
  $FUN~3$ & $1.77e-1$ & $1.98e-1$ & $8.62e-1$ & $5.41e-1$ \\      \hline
  $FUN~4$ & $3.04e-1$ & $1.48e-1$ & $8.87e-1$ & $6.21e-1$ \\      \hline
  $FN~1$ & $1.49e-1$ & $1.63e-2$ & $9.33e-1$ & $5.75e-1$ \\      \hline
  $FN~2$ & $1.46e-1$ & $1.91e-1$ & $9.71e-1$ & $7.82e-1$ \\      \hline
  $FN~3$ & $1.16e-1$ & $2.01e-1$ & $9.78e-1$ & $8.68e-1$ \\      \hline
    $FN~4$ & $1.40e-1$ & $1.50e-1$ & $9.88e-1$ & $9.08e-1$ \\      \hline
    
\end{tabular}
\end{table}

 \textbf{Issue of Ill-Conditioning:} In the case of \emph{tile} coding the matrix $\E{\phi{\phi'}^\top}$ has a lot of eigenvalues which are either $0$ or have very small magnitude, due to which we could not invert the $A$ matrix and $\ts$ is not available. The effect was these small/$0$ eigenvalues were reflected in the convergence as well. In both TD(0) and GTD, while the iterates did not blow exponentially (as it is in the case of instability) they kept drifting away from the origin, and, in the case of TD(0) the estimation error reduced, and with GTD the estimation error in certain cases did not reduce less than $0.5$ and  hence not reported as $NA$ the \Cref{fig:results}. 
 
 \textbf{Behaviour when conditioning is good:} However, in the case of Fourier basis functions (normalized/un-normalized) both TD(0)/GTD performed well in \emph{on/off}-policy settings. In general, we observe that the GTD is slower than TD and the \emph{off}-policy was slower than \emph{on}-policy. These can be seen by the fact that we need around $2e4$ for \emph{on}-policy and $5e5$ for \emph{off}-policy and in the case of GTD in \emph{off}-policy we observed that complete convergence happens only after $5e6$ iterations (the tables in \Cref{fig:results} report only $5e5$ so that GTD and TD(0) can be compared). 
\end{comment}