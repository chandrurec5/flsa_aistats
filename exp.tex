%!TEX root =  flsa.tex
\section{Numerical Experiments}\label{sec:exp}

\textbf{Worst-case error;}
The goal here was to illustrate \cref{th:pspd}, which 
proved results for the behavior
of the worst-case errors $\eps_t(\SPD)$, $\eps_t'(\SPD)$, and $\eps_t'(\SPDSN)$.
To validate this result, we chose $d=2$ and define the classes $\text{USN}$ (``unscaled noise'') and $\text{SN}$ (``scaled noise'') as subsets of $\SPD$ and $\SPDSN$, respectively. 
To define these classes let $\{u_t\}_{t\ge 1}\subset \R^2$ be in i.i.d. sequence so that $u_{t,1}$ and $u_{t,2}$ are also independent and they are both uniformly distributed in $[-1,1]$.
Now, $P$ is in $\text{USN}$ when $A_P=\left[\begin{matrix}1 &0\\ 0 & a_P\end{matrix}\right]$ for some $a_P\in (0,1)$,
$A_t = A_P$ for all $t\ge 1$ and $b_t = u_t$.
Further, $P$ is is in $\text{SN}$ when $A_P$ and $\{A_t\}_{t\ge 1}$ are as in $\text{USN}$ and $b_t = A_P u_t$.
%Note that $\text{SN}\subset \SPDSN$ and $\text{USN}\subset \SPD$.
The upper left subfigure in \Cref{fig:results} shows lower bounds on $\eps_t(\text{USN})$, $\eps_t'(\text{USN})$, 
and $\eps_t'(\text{SN})$ as a function of the number of rounds, or iterations.
The stepsize for producing $\eps_t(\text{USN})$ is chosen to be $\alpha = 0.9$, while to obtain a lower bound on $\eps_t(\text{USN})$ we let $a_P=1/t$. We can observe that the lower bound increases linearly with $t$.
For producing a lower bound on $\eps_t'(\text{USN})$, we let $a_P = 1/\sqrt{t}$ and also $\alpha = 1/\sqrt{t}$.
Observe that the lower bound decreases as $1/\sqrt{t}$, as expected.
Finally, to produce a lower bound on $\eps_t'(\text{SN})$, we chose $\alpha = 0.9$ and $a_P = \frac{1}{t}$. 
The lower bound decreases as $1/t$, as expected.

\textbf{Mountain Car (setup):}  The mountain car is a widely used domain for illustrating control learning in RL.
However, here, we use it for illustrating linear value estimation only. The domain consists of an underpowered car, that needs to swing from the bottom of a valley to the top by performing either one of the three possible actions: \emph{forward, reverse, no} throttle. Since the car is underpowered, it cannot directly accelerate to the top from the bottom and needs to swing back and forth to reach the top. The state of the system is described by the position $p$ and the velocity $v$ of the car at a given time. For the purpose of \emph{on-policy} evaluation, we sample from the policy $\pi$ that accelerates in the direction of the velocity with probability $\frac{298}{300}$ and the other two actions with probability $\frac{1}{300}$ each. Since, we are also interested in the \emph{off-policy} case, we sampled using a behavior policy $\pi_b$ that accelerates in the direction of the velocity with probability $\frac{8}{10}$ and chooses the other two actions with probability $\frac{1}{10}$ each. We used \emph{tile coding} and \emph{Fourier} basis (un-normalized and normalized). We used $4$ different tiling ($4\times 4$ and $7\times 7$ grid for the two state-variables permuted with $5$ and $10$ tiles), and we also tried $4$ different $n^{th}$ Fourier basis function ($n=3,5,7,9$), with $d=(n+1)^2$. For a given state $s=(p,v)$,\footnote{We scale the states by subtracting the minimum value and dividing it by its range, so that $p,v\in(0,1)$ after scaling.} the Fourier feature is given by $\phi(p,v)=\big(cos(\pi [c_1p+c_2 v],c_1,c_2=0,1,\ldots,n)\big) \in \R^{d}$, where $d=(n+1)^2$. The normalized features were obtained by letting $\normsm{\phi(s)}^2_2=1$. We generated $100$ trajectories for the \emph{on/off}-policies, and the discount factor we used was $\gamma=0.999$. 

Before discussing the observations, we digress, to mention two important aspects related to LSA algorithms, which, while being out of the scope of this paper, nevertheless are important in practice.

\textbf{Singularity:} In \Cref{assmp:lsa} we assumed that the matrix $A_P$ is Hurwitz and hence invertible. When the underlying matrix is singular, there could be two scenarios: either $A_P\theta=b_P$ has infinitely many solutions, or it has no solutions. In the former scenario, and under a further assumption that the null-space of $A_P$ is diagonalizable (see \cite{bertstab}), the null space can be discarded after applying an appropriate linear transformation $U$ (as in \Cref{th:rate}) to a obtain a reduced linear system $\tilde{A}_P\tilde{\theta}=\tilde{b}_P$. This reduced linear system has a unique solution and then the results of \Cref{th:rate} can be applied.

\textbf{Design of Updates:} Note that the CATD(0) and CAGTD have different underlying linear systems. This is evident by writing down $(b_P,A_P)$ for TD(0) and GTD respectively. Let $A_t=\phi_t(\phi_t-\gamma\phi'_t)^\top$, $b_t=r_t \phi_t$.
Then, for CATD(0), $A_{TD}=\EE{A_t}$ and $b_{TD}=\EE{b_t}$.  For CAGTD we have $A_{GTD}=\left[\begin{matrix}I & A_{TD} \\ -(1-\alpha )A_{TD}^\top &\alpha A_{TD}^\top A_{TD}\end{matrix}\right]$ and $b_{GTD}=[b_{TD}^\top,\alpha b_{TD}^\top A_{TD}]^\top$. For CAGTD, the eigenvalues involve $A_{TD}^\top A_{TD}$, i.e., a small eigenvalue of $A_{TD}$ gets squared.
Consequently CAGTD can be poorly conditioned compared to CATD(0).
%!TEX root =  flsa.tex
\begin{figure}
\begin{tabular}{cc}
\begin{minipage}{0.45\columnwidth}
 \resizebox{1\columnwidth}{!}{
\begin{tikzpicture} 
\begin{loglogaxis}
    [
      error bars/y dir      = both,
      error bars/y explicit = true,
      legend pos=north west,
      xlabel=iters,
      ylabel=$\log( MSE)$
    ]
        \addplot[only marks, mark=+] table[y ]  {est/est};
    \addplot[only marks, mark=o] table[y ]  {pred/pred};
        \addplot[only marks, mark=*] table[y ]  {struc/struc};
\addlegendentry{$\eps_t(\text{USN})$}
\addlegendentry{$\eps_t'(\text{USN})$}
\addlegendentry{$\eps_t'(\text{SN})$}
%\addplot[smooth] table[y error index = 2]  {tdon/tile1};    

\end{loglogaxis}
 \end{tikzpicture}
  }
 \end{minipage}
&


\begin{minipage}{0.45\columnwidth}
 \resizebox{1\columnwidth}{!}{
\begin{tikzpicture} 
\begin{axis}
    [
      error bars/y dir      = both,
      error bars/y explicit = true,
      legend pos=north west,
      xlabel=iters,
      ylabel=$\EE{\norm{\thh_t}^2}$
    ]
        \addplot[only marks, mark=+] table[y ]  {baird/gtd};
    \addplot[only marks, mark=o] table[y ]  {baird/gtd_cs};
\addlegendentry{$\alpha=0.005,\beta=16$}
\addlegendentry{$\alpha=\frac{1}{4},\beta=1$}
%\addplot[smooth] table[y error index = 2]  {tdon/tile1};    

\end{axis}
 \end{tikzpicture}
  }
 \end{minipage}

\\

\begin{minipage}{0.45\columnwidth}
 \resizebox{1\columnwidth}{!}{
\begin{tikzpicture} 
\begin{axis}
    [
      error bars/y dir      = both,
      error bars/y explicit = true,
      legend pos=north west,
      xlabel=iters $\times 2e^2$,
      ylabel=$\EE{\norm{A_{TD}\thh_t-b_{TD}}^2}$
    ]
        \addplot[smooth, dotted] file  {exp_camera/tdon/tdtile};
    \addplot[smooth, dashed] file  {exp_camera/tdon/tdfouriern1};
        \addplot[smooth] file{exp_camera/tdon/tdfourierun1};
\addlegendentry{T}
\addlegendentry{FN}
\addlegendentry{FUN}
%\addplot[smooth] table[y error index = 2]  {tdon/tile1};    
\end{axis}
 \end{tikzpicture}
  }
 \end{minipage}

&
\begin{minipage}{0.45\columnwidth}
 \resizebox{1\columnwidth}{!}{
\begin{tikzpicture} 
\begin{axis}
    [
      error bars/y dir      = both,
      error bars/y explicit = true,
      legend pos=north west,
      xlabel=iters $\times 2e^2$,
      ylabel=$\EE{\norm{A_{GTD}\hat{z}_t-b_{GTD}}^2}$
    ]
        \addplot[smooth, dotted] file  {exp_camera/gtdon/gtdtile_gh};
    \addplot[smooth, dashed] file  {exp_camera/gtdon/gtdfouriern1_gh};
        \addplot[smooth]	 file	{exp_camera/gtdon/gtdfourierun1_gh};
\addlegendentry{T}
\addlegendentry{FN}
\addlegendentry{FUN}
%\addplot[smooth] table[y error index = 2]  {tdon/tile1};    
\end{axis}
 \end{tikzpicture}
  }
 \end{minipage}

 
 \\
 
 
 \begin{minipage}{0.45\columnwidth}
 \resizebox{1\columnwidth}{!}{
\begin{tikzpicture} 
\begin{axis}
    [
      error bars/y dir      = both,
      error bars/y explicit = true,
      legend pos=north west,
      xlabel=iters$\times 2e^2$ for CATD(0) (iters $\times 2e^3$ for CAGTD),
      ylabel=$\EE{\norm{A_{TD}\thh_t-b_{TD}}^2}$
    ]

        \addplot[smooth,dashed] file{exp_camera/gtdon/fourierun1_errlin};
         \addplot[smooth] file{exp_camera/tdon/tdfourierun1};

\addlegendentry{CATD(0)}
\addlegendentry{CAGTD}

%\addplot[smooth] table[y error index = 2]  {tdon/tile1};    
\end{axis}
 \end{tikzpicture}
  }
 \end{minipage}

&
\begin{minipage}{0.45\columnwidth}
 \resizebox{1\columnwidth}{!}{
\begin{tikzpicture} 
\begin{axis}
    [
      error bars/y dir      = both,
      error bars/y explicit = true,
      legend pos=north west,
       xlabel=iters$\times 2e^2$,
      ylabel=$\EE{\norm{A_{TD}\thh_t-b_{TD}}^2}$
    ]
        \addplot[smooth,dotted] file{exp_camera/gtdon/tile1_errlin};
                    \addplot[smooth,dashed] file{exp_camera/gtdon/fouriern1_errlin};
            \addplot[smooth] file{exp_camera/gtdon/fourierun1_errlin};

\addlegendentry{T}
\addlegendentry{FN}
\addlegendentry{FUN}
%\addplot[smooth] table[y error index = 2]  {tdon/tile1};    
\end{axis}
 \end{tikzpicture}
  }
 \end{minipage}

\end{tabular}
%\caption{Summary of experiments. The top left plot is the experiment that illustrates \Cref{th:pspd} on uniform rates. The top right plot is the result for BAIRD domain. The rest of the plots are correspond to experiments in CATD($0$) and CAGTD, here, T, FUN, FN stand for tile coding, Fourier un-normalized and normalized respectively. The plots in the second row show the MSE in the case of CATD($0$) and CAGTD, where $\hat{z}_t=[\hat{y}_t^\top, \thh_t^\top]^\top \in \R^{2d}$ contains both dual and primal variables of the CAGTD algorithm. The left most plot in the third row shows TD error for CATD(0) and CAGTD algorithms, note that only the primal variable $\thh_t$ of CAGTD enters the error term $\norm{A_{TD}\thh_t-b_{TD}}$. The right plot in the third row shows the convergence of the TD error in the case of CAGTD algorithm for the three types of basis. In the plots in the second and the third row, the $x$-axis is number of iterations, with $2e^4$ as the maximum. However, the CAGTD curve in the bottom left has been made $10\times$ faster, i.e., the maximum number of iterations is $2e^5$.} 
\caption{\small Experimental results. T, FUN, FN stand for tile coding, Fourier un-normalized and normalized respectively.  In the bottom left plot, the CAGTD curve has been made $10\times$ faster, i.e., the $x$-axis runs till $2e^5$ iterations and for CATD(0) the $x$-axis runs till $2e^4$ iterations.} 
\label{fig:results}
\end{figure}




\begin{itemize}[leftmargin=*]
\item \textbf{Stability:} For CATD(0), we ran  \emph{on-policy} with all the three features and \emph{off-policy} with normalized features. For CAGTD, we ran with all the features and both \emph{on/off-policy}. In all the experiments, we chose the stepsize dictated by \Cref{th:tdadmis}. All the experiments were stable (bottom two rows of \Cref{fig:results}). The values are averaged over $10$ runs \todoc{$10$ seems to be a small number.} and since the variance was observed to be small, 
to reduce clutter, error bars are not shown.
\item \textbf{Near Singularity: } We observed in the case of tile coding and normalized Fourier basis functions the underling $A_{TD}$ matrices were nearly singular, i.e., small eigenvalues with positive real-parts close $0$. However, we observed that the error $\EE{\normsm{A_{TD}\thh_t -b_{TD}}^2}$ converges to $0$ (left plot in the second row of \Cref{fig:results}). We also observed that $\EE{\normsm{A_{GTD}\hat{z}_t-b_{GTD}}^2}$ converges to $0$ (right plot in the second row of \Cref{fig:results}), where $\hat{z}_t=[\hat{y}_t^\top,\thh_t^\top]^\top$. Here, $\thh_t$ is the primal variable and $\hat{y}_t$ is the dual variable. However, for CAGTD, $\EE{\normsm{A_{TD}\thh_t-b_{TD}}^2}$ does not always converge to $0$ (tile coding in right plot in the third row of \Cref{fig:results}). This might be due to the fact that linear systems underlying CATD(0) and CAGTD are different. 
\item \textbf{Slowness of GTD:} Unnormalized Fourier basis were better conditioned in comparison to the other basis choices. In this case, for CATD(0) and CAGTD $\EE{\normsm{A_{TD}\thh_t -b_{TD}}^2}$ converges to $0$. However,  CAGTD is slower in comparison to CATD(0) (left plot in the third row of \Cref{fig:results}). \todoc{You wrote: ``due to the fact that $A^\top_{TD}A_{TD}$ matrix is involved in the spectrum of CAGTD, leading to the squaring of small eigenvalues.'' Fine, but this has been said before. Also, can't we actually show the spectrum? It would be more convincing to show those eigenvalues than just talking about them.}
\end{itemize}

\textbf{BAIRD:} In this domain there are $S=\{s_1,\ldots,s_7\}$ states and $A=\{a_1,a_2\}$ actions. Under, $a_1$ we have $p_{a_1}(s,s_1)=1$ for all $s\in S$ and under $a_2$ we have $p_{a_2}(s,s')=\frac{1}{6}$ for all $s\in S, s'=2,\ldots,7$. The samples are collected using a behaviour policy $\pi_b$ that performs action $a_2$ with probability $\frac{6}{7}$ and action $a_1$ with probability $\frac17$, and the target policy that we are interested is $\pi$ which performs action $a_1$ in all the states. The feature vector we chose was: $\phi(s_1)=[\frac{1}{2}\,0\,0\,0\,0\,0\,0\,1]$, $\phi(s_i)=e_i+[0\,0\,0\,0\,0\,0\,0\,\frac{1}{2}], i=2,\ldots,7$, where $e_i$ is the standard basis with $i^{th}$ co-ordinate $1$ and rest of the co-ordinates $0$. Since $\phi'_t$ always corresponds to state $1$ and is different from $\phi_t$, in this example $\E[{\phi_t\phi_t}]\neq\E[{\phi'_t{\phi'_t}^\top}]$.
We compared the performance of CAGTD with $\alpha=0.005$ (and $\beta=16$, see \cite{gtdmp}) with the choice of $\alpha=\frac{1}{2\times 2}$ ($2$ is to normalize the features) and initial condition $\theta_0=[1\, 1\, 1\, 1\, 1\, 1\, 10\, 1]$. The identical stepsize of $\frac{1}{4}$ performed better than choosing different stepsizes for the primal and dual variables. Please refer to the top right plot of \Cref{fig:results}. \todoc{The figure label suggest that $\alpha = 1/4$, $\beta = 1$. So no identical stepsizes!? I am really confused.}

