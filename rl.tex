%!TEX root =  flsa.tex
\section{Universal stepsizes in LVE}\label{sec:rl}
We now turn to the question of the existence of universal stepsizes for CATD(0) and CAGTD (cf. \cref{tb:tdalgos}). \todoc{Extensions to TD($\lambda$) etc.?}
In what follows, we define what we call \emph{admissibility}, 
a sufficient condition for the existence of a universal stepsize. 
\begin{definition}\label{def:admis}
Call a problem class $\P$ \emph{admissible} if there exists a unique $U$ and $\alpha_{\P_U}>0$ such that
$\rhos{P_U}>0$ holds for all $P\in \P$ and $\alpha\in(0,\alpha_{\P_U})$.
\end{definition}
If $\P$ is admissible, it  follows from \Cref{th:rate} that an asymptotic ``fast'' rate of $O(\frac1t)$ is achieved for any $P\in \cP$. 
We now define three LVE problem classes. For the definitions introduce
the entrywise $\max$-norm for matrices: $\norm{A}_{\max} = \max_{i,j} |A_{ij}|$.
Recall that an LVE problem is given by the joint distribution of the i.i.d. sequence
$\{(\phi_t,\phi_t',r_t)\}_{t\ge 1}\subset \R^d\times \R^d \times \R$. We let 
$A_t^\delta = \phi_t(\phi_t - \delta \phi_t')^\top$ for $\delta\in (0,1)$.
The classes we define satisfy $\norm{A_t}_{\max}\le B$ with a known constant $B>0$, 
as well as the following conditions:
\begin{align*}
\text{TDON}(B)&: \E[\phi_t\phi_t^\top]=\E[\phi'_t(\phi'_t)^\top], \\
						& \,\,\,\,\, \EE{A_t^\delta} \text{ is PD for all } \delta\in (0,1)\,;\\
\text{TDOFF}(B)&: \E[\phi_t\phi_t^\top]=\E[\phi'_t(\phi'_t)^\top], \norm{\phi_t}=1\,;\\
\text{GTDOFF}(B)&: \text{no extra condition}
\end{align*}
The rationale of the class names should become clear after the next result.
Here we note for the reader familiar with the RL literature that the classname ending of ``on'' vs. ``off'' signifies
whether the condition holds in the so-called ``on-policy'', or the ``off-policy'' case, respectively.
Note that a classical result establishes that $\EE{A_t^\delta}$ is indeed PD for the ``on-policy'' case
\cite{su88},
while the ``second order feature stationarity'' condition $ \E[\phi_t\phi_t^\top]=\E[\phi'_t(\phi'_t)^\top]$ will hold 
when sampling (in the underlying Markov reward process) 
is started from the stationary distribution \cite{su88}. 
\if0
\FloatBarrier
\begin{table}[h]
\resizebox{\columnwidth}{!}{
\begin{tabular}{|c|c|c|c|}\hline
Class  &Variable &Remark\\\hline
$\P_{TDON}$  & $\pi,P,R$ &$\mu=d_{\pi}, \EE{\phi_t\phi_t}=\EE{\phi'_t\phi'_t}$\\\hline
$\P_{TDOFF}$  & $\pi,\mu,P,R$ &$\mu\neq d_{\pi}, \EE{\phi_t\phi_t}=\EE{\phi'_t\phi'_t}$, $\norm{\phi_t}=1$\\\hline
$\P_{GTDOFF}$ & $\pi,\mu,P,R$ &$\mu\neq d_{\pi}$\\\hline
\end{tabular}
}
\caption{Here, $S,A,\gamma$ are fixed across all the class and the second column shows the quantities that vary across the respective classes. These three capture \emph{on/off-policy} learning scenarios arising in RL.}
\end{table}
\fi
\begin{restatable}{theorem}{thtdadmis}\label{th:tdadmis}
The following hold:
%Define constant $B\eqdef\max_{ij}\md{A_{ij}}$.  We have 
$i)$ CATD(0) has a universal stepsize of $\alpha_{td}=\frac{1}{B^2d}$ for  the class $\text{TDON}(B)$.
$ii)$ CATD(0) has a universal stepsize of $\alpha_{td}=1$ for  the class $\text{TDOFF}(B)$.
$ii)$ CAGTD has a universal stepsize of $\alpha_{gtd}=\frac{1}{2B^4d^2}$ for the class $\text{GTDOFF}(B)$.
\end{restatable}
In the proof we show that the three classes are admissible for the respective algorithms.
\if0
The property $\EE{\phi_t\phi_t}=\EE{\phi'_t{\phi'_t}^\top}$ is known as second order feature stationarity, and is used in the proof.
\fi

 From \Cref{tb:tdalgos}, the matrix $A_t=\phi_t(\phi_t-\gamma\phi'_t)^\top$ is key to both CATD(0) and CAGTD. In the case of CATD(0),  the expression for $\rhos{P}$ involves $\phi^\top_t\phi_t$ which can take a maximum value of $B^2d$  and is the reason for the $B^2d$ term in $\alpha_{td}$. In the case of CAGTD, the expression for $\rhos{P}$ involves $A_t^\top A_t$, and hence $B^4d^2$ appears in $\alpha_{gtd}$. 
The conservative stepsize for CAGTD seems to be the price paid for \emph{off-policy} stability. Notice that in the case of normalized features the factor depending on $B$ and $d$ can be removed.

Note that the above result in particular implies that the respective algorithms with the proposed stepsizes achieve the instance-dependent errors $O(\frac{1}{t})$ on these three classes of LVE problems.

