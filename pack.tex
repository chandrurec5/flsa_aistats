%!TEX root =  flsa.tex
%!TEX encoding = UTF-8 Unicode
%Packages
\usepackage{hyperref}
\usepackage{amsmath}
\usepackage{amsthm,amssymb}
\usepackage{comment}
\usepackage{amsfonts}
\usepackage{graphicx}
\usepackage{multirow}% http://ctan.org/pkg/multirow
%\usepackage{ntheorem}
%\usepackage[textsize=tiny]{todonotes}
\usepackage{tikz}
\usepackage{pgfplots}
\usepackage{enumitem}
\usepackage{algorithm}
\usepackage{algorithmic}
\usepackage{pifont}
\usepackage{makecell}
\usepackage{slashbox}
%\usepackage[numbers, sort, comma, square]{natbib}
%New Commands
\DeclareMathOperator{\supp}{supp}
\newcommand{\ip}[1]{\langle#1\rangle}
\newcommand{\norm}[1]{\left\| #1\right\|}
\newcommand{\normsm}[1]{\| #1\|}
\newcommand{\lmin}[1]{\lambda_{\min}\left(#1\right)}
\newcommand{\lmax}[1]{\lambda_{\max}\left(#1\right)}
\newcommand{\eqdef}{\stackrel{\cdot}{=}}
\setlength{\marginparwidth}{13ex}
\newcommand{\todoc}[2][]{\todo[size=\scriptsize,color=blue!20!white,#1]{Csaba: #2}}
\newcommand{\todoch}[2][]{\todo[size=\scriptsize,color=red!20!white,#1]{Chandru: #2}}
\usepackage[disable]{todonotes}
\usepackage{todonotes}
\usepackage{placeins}
\usepackage{xspace}

\newcommand{\vh}{\hat{V}_\pi}
\newcommand{\snorm}[1]{\left\|#1\right\|}
\newcommand{\dcd}{d \times d}
\newcommand{\err}{\emph{err}}
\newcommand{\B}{\mathcal{B}}
\newcommand{\cP}{\mathcal{P}}
\newcommand{\V}{\mathcal{V}}
\newcommand{\nn}{\nonumber}
\newcommand{\cond}[1]{\kappa(#1)}
\newcommand{\md}[1]{\left|#1\right|}
\newcommand{\rhod}[1]{\rho_d(\alpha,#1)}
\newcommand{\rhos}[1]{\rho_s(\alpha,#1)}
\newcommand{\alphaps}{\alpha_s(P)}
\newcommand{\alphapd}{\alpha_d(P)}
\newcommand{\ra}{\rightarrow}
\newcommand{\zero}{\mathbf{0}}
\renewcommand{\P}{\mathcal{P}}
\newcommand{\pc}{p_{\mathcal{c}}}
\newcommand{\E}{\mathbf{E}}
\newcommand{\F}{\mathcal{F}}
\newcommand{\R}{\mathbb{R}}
\newcommand{\T}{\mathcal{T}}
\newcommand{\EE}[1]{\mathbf{E}\left[#1\right]}
\newcommand{\EEP}[1]{\mathbf{E}_P\left[#1\right]}
\newcommand{\eep}[2]{\mathbf{E}_{#2}\left[#1\right]}
\newcommand{\gln}{\mathrm{GL}(d)}
\newcommand{\gld}{\mathrm{GL}(d)}
\newcommand{\ncn}{n\times n}
\newcommand{\I}{\mathcal{I}} % something better?
\newcommand{\C}{\mathbb{C}}
\newcommand{\re}[1]{\emph{re}(#1)}
\newcommand{\im}[1]{\emph{im}(#1)}
\newcommand{\op}{\oplus}
\newcommand{\tL}{\tilde{\Lambda}}
\newcommand{\tJ}{\tilde{J}}
\newcommand{\Lt}{\Lambda_t}
\newcommand{\ts}{\theta_*}
\newcommand{\eb}{\bar{e}}
\newcommand{\tb}{\bar{\theta}}
\newcommand{\zb}{\bar{z}}
\newcommand{\zh}{\hat{z}}
\newcommand{\eh}{\hat{e}}
%\newcommand{\rhoD}{\rho_D(\alpha,U,A)}
%\newcommand{\rhoR}{\rho_R(\alpha,U,A)}
\newcommand{\SD}{S^D_{\alpha,U}}
\newcommand{\SR}{S^R_{\alpha,U}}
\newcommand{\bu}{\beta(\alpha,U,P)}
\newcommand{\thh}{\hat{\theta}}
\newcommand{\gh}{\hat{\gamma}}
\newcommand{\iid}{\emph{i.i.d.}\xspace}

\newcommand{\M}{\mathcal{M}}
\renewcommand{\P}{\mathcal{P}}
\renewcommand{\S}{\mathcal{S}}
%\renewcommand{\R}{\mathcal{R}}
\newcommand{\A}{\mathcal{A}}
\newcommand{\defeq}{\doteq}

%Theorem Definition
\theoremstyle{definition}
\newtheorem{theorem}{Theorem}
\newtheorem{example}{Example}
\newtheorem{remark}{Remark}
\newtheorem{domain}{Domain}
\newtheorem{condition}{Condition}
%\newtheorem{definition}{Definition}
\newtheorem{corollary}{Corollary}
\newtheorem{lemma}{Lemma}
\newtheorem{proposition}{Proposition}
%\theoremstyle{assumption}
%\newtheorem{assumption}{Assumption}

%%%%%%%%%%%%%%%%%%%%%%%%%%%%%%%%%%%%%%%%%%%%%%%%
% Theorem environments and cleveref
%%%%%%%%%%%%%%%%%%%%%%%%%%%%%%%%%%%%%%%%%%%%%%%%
%\newtheorem{assumption}{Assumption}
%\usepackage[capitalize]{cleveref}
\crefname{assumption}{Assumption}{Assumption}
\usepackage{thmtools}
\usepackage{thm-restate}

\let\lemma\relax
\declaretheorem[name=Lemma,refname={Lemma,Lemmas},Refname={Lemma,Lemmas},sibling=theorem]{lemma}

\let\corollary\relax\declaretheorem[name=Corollary,refname={Corollary,Corollaries},Refname={Corollary,Corollaries},sibling=theorem]{corollary}

\declaretheorem[name=Assumption,refname={Assumption,Assumptions},Refname={Assumption,Assumptions}]{assumption}

\let\proposition\relax\declaretheorem[name=Proposition,refname={Proposition,Propositions},Refname={Proposition,Propositions},sibling=theorem]{proposition}

\let\definition\relax
\declaretheorem[name=Definition,refname={Definition,Definitions},Refname={Definition,Definitions}]{definition}

% \crefname{question}{question}{questions}
\Crefname{question}{Question}{Questions}
\creflabelformat{question}{(#2#1#3)}

% \crefname{problem}{problem}{problems}
\Crefname{problem}{Problem}{Problems}
\creflabelformat{problem}{(#2#1#3)}

% \crefname{equation}{equation}{equations}
\Crefname{equation}{Equation}{Equations}
\creflabelformat{equation}{(#2#1#3)}

%\crefname{iCondition}{\ccap{c}ondition}{\ccap{c}onditions}
\crefname{iCondition}{Condition}{Conditions}
\creflabelformat{iCondition}{(#2#1#3)}
\crefrangelabelformat{iCondition}{(#3#1#4) to (#5#2#6)}

\Crefname{item}{}{}
\creflabelformat{item}{(#2#1#3)}
\crefrangelabelformat{item}{(#3#1#4) to (#5#2#6)}

%%%%%%%%%%%%%%%%%%%%%%%%%%%%%%%%%%%%%%%%
