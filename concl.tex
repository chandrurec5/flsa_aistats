%!TEX root =  flsa.tex
\textbf{Conclusion:} We showed that the CALSA achieves an instance dependent fast rate of $O(\frac{1}{t})$, and certain interesting problem classes admit a universal step-size. The results are exciting in the context of TD(0) and GTD (where prior rate of $O(\frac{1}{\sqrt{t}})$ was known), given the fact that no projection to compact sets were used (say as in \cite{gtdmp}). There is also some ongoing work in TD(0) in the literature, however, as \cite{issues}  points out there are some technicalities that need to addressed as well. The results in the experiments were also promising in that the proposed constant step-size was a stable choice. The important open question concerning uniforn rate is open (for RL as well as linear prediction in the general multiplicative case). We believe understanding the relationship between the Gram matrix $\E[\phi_t\phi_t^\top]$ and the matrix $\E[\phi_t(\phi_t-\gamma{\phi'_t})^\top]$ is  key to obtain extra structural insights.