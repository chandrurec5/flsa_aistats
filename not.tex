%!TEX root =  flsa.tex
\textbf{Notation:} The set of reals is denoted by $\R$. 
%For $x\in \C$ we denote its modulus and complex conjugate by $\md{x}$ and $\bar{x}$, respectively. 
The $d$-dimensional vector space over $\R$ is denoted by $\R^{d}$,
while $\R^{p\times q}$ denotes the vector space of $p\times q$ matrices over the reals.
Vectors are column vectors (i.e., $\R^d$ is identified with $\R^{d\times 1}$).
The transpose of a matrix $C$ is denoted by $C^\top$
(and of course the same notation applies to vectors, as well). 
We will use $\ip{\cdot,\cdot}$ to denote inner products: $\ip{x,y}=x^\top y$
and use $\norm{x} = \ip{x,x}^{1/2}$ to denote the $2$-norm.
We call a matrix $A\in \R^{\dcd}$  \emph{Hurwitz} (H) if all eigenvalues of $A$ have strictly positive real parts. 
We call a matrix $A\in \R^{\dcd}$ \emph{positive definite} (PD) if $\ip{x,Ax} >0$ for all nonzero $x \in \R^{d}$.
If $\inf_x \ip{x,Ax}\ge 0$ then $A$ is \emph{positive semi-definite} (PSD).
We call a matrix $A\in \R^{\dcd}$ to be \emph{symmetric positive definite} (SPD) if 
it is symmetric i.e., $A^\top=A$ and PD. 
For $C\in \R^{\dcd}$ SPD and $x\in \R^d$, we let $\norm{x}^2_C\eqdef x^\top C \,x$.
The spectral norm of the matrix $A$ is given by $\norm{A}\eqdef \sup_{x\in \R^d:\norm{x}=1} \norm{Ax}$.  
The spectral radius of $A$ is $\rho(A)\defeq \max\{ |\lambda|\,:\, \lambda \in \Lambda(A) \}$ where $\Lambda(A)$ is the set of (complex) eigenvalues of $A$. For symmetric matrices $\rho(A) = \norm{A}$.
We use $\cond{A}=\normsm{A}\normsm{A^{-1}}$ to denote the condition number of a non-singular matrix $A$. 
We use $\lmin{A}$ to denote the minimum eigenvalue of a symmetric matrix $A$, while $\lmax{A}$ denotes its maximum eigenvalue.
%We denote the spectral radius of a matrix $A$ by $\Lambda(A)=\{\max_i \left|\lambda_i\right|: \lambda_i ~\text{is an eigenvalue of}~A\}$.
We denote the identity matrix in $\R^{\dcd}$ by $I$. 
%and the set of invertible $\dcd$ complex matrices by $\gld$. \todoc{Is this used ever?}
%For a positive real number $B>0$, we define $\C^{d}_B=\{b\in \R^d\mid \norm{b}\leq B\}$ and $\C^{\dcd}_B=\{A\in \R^{\dcd}\mid \norm{A}\leq B\}$ to be the balls in $\R^d$ and $\R^{d\times d}$, respectively, of radius $B$.  \todoc{Are these used?!}
We use $Z\sim P$ to denote the fact that $Z$ (which can be a number, or vector, or matrix) is distributed according to probability distribution $P$; $\E$ denotes mathematical expectation. 
For $t\ge 0$ let $a_t,b_t:X \to (0,\infty)$. We write $a_t\asymp b_t$ 
when there exists constants $0\le c_1\leq c_2$ such that for any $x\in X$, $c_1 a_t(x)\leq b_t(x)\leq c_2 a_t(x)$. 
%When $a_t,b_t$ depend on constants, $c_1$ and $c_2$ do not depend on those.
\begin{comment}
Let us now state some definitions that will be useful for presenting our main results. \todoc{Actually, we should remove all definitions not needed by the main body.}
\begin{definition}\label{def:dist}
For a probability distribution $P$ over $\C^d \times \C^{d\times d}$, we let $P^V$ and $P^M$ 
denote the respective marginals of $P$ over $\C^d$ and $\C^{d\times d}$. \todoc{Changed this.}
By \emph{abusing notation} we will often write $P = (P^V,P^M)$ to mean that $P$ is a distribution with the given marginals.
%Let $P=(P^V,P^M)$ denote a $2$-tuple of probability distributions; $P^V$ over $\C^{d}$ and $P^M$ over $\C^{\dcd}$. 
Define $A_P=\int M\, dP^M(M), C_P=\int M^* M \,dP^M(M),  b_P=\int v\, dP^V(v)\,,
\rhod{P}\eqdef {\inf}_{x\in\C^d\colon\norm{x}=1}\ip{x,\left((A_P+A_P^*)-\alpha A_P^* A_P\right)x},
\rhos{P}\eqdef{\inf}_{x\in \C^d\colon\norm{x}=1}\ip{x,\left((A_P+A_P^*)-\alpha C_P\right)x}\,.
$
\end{definition}
Note that $\rhod{P}\ge \rhos{P}$. Here, subscripts $s$ and $d$ stand for \emph{stochastic} and \emph{deterministic} respectively.  \todoc{Strict inequality?? How about $P^M$ concentrating on zero? I changed the strict inequality to non-strict.}
\todoc{Explain why we use subindex $d$ and $s$.}
\todoc{Since $\rhod$ depends on $P^M$ only, why not make it a function of $P^M$ only? Or at least add a remark?}
\begin{definition}\label{def:simdist}
Let $P=(P^V,P^M)$ as in \Cref{def:dist}; $b\sim P^V$ and $A\sim \P^M$ be random variables distributed according to $P^V$ and $P^M$. For $U\in \gld$ define $P_U$ to be the  distribution of $(U^{-1}b,U^{-1}AU)$. We also let
$(P_U^V,P_U^M)$ denote the corresponding marginals. \todoc{I hope this works out.}
\end{definition}
\begin{definition}
\end{definition}
Note that SPD implies that the underlying matrix is real.
\begin{definition}\label{distpd}
We call the distribution $P$ in \Cref{def:dist} to be H/PD/SPD if $A_P$ is H/PD/SPD.
\end{definition}
Though $\rhos{P}$ and $\rhod{P}$ depend only on $P^M$, we use $P$ instead of $P^M$ to avoid notational clutter.
\end{comment}