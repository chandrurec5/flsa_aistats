
%!TEX root =  flsa.tex
\section{Introduction}\label{sec:intro}
Various estimation and filtering problems in signal processing, reinforcement learning or econometrics
are formulated as the problem of finding the unique solution $\ts\in \R^d$ 
to the linear equation $\EE{A_t} \theta = \EE{b_t}$,
where $\{(A_t,b_t)\}_{t\ge 1}$ is an $\R^d \times \R^{\dcd}$-valued random sequence with a common distribution $P$
and the expectation $\EE{A_t}$ of the matrix $A_t$ is non-singular \citep[e.g.,][]{bemepri90,LjPfWa92,SoKo94,degylu96,
sutton,konda-tsitsiklis,KoTsi03LSA,gtd,gtd2,gtdmp}.
Oftentimes, the matrices $A_t$ are rank-1, $\EE{A_t}$ is Hurwitz (its eigenvalues have positive real parts)
and the dimensionality $d$ is large.
Then, for any positive-valued, user-chosen stepsize sequence $\{\alpha_t\}_{t\ge 1}$, the updates
\begin{align}\label{eq:lsaintro}
\theta_t=\theta_{t-1}+\alpha_t (b_t-A_t \theta_{t-1})
\end{align}
can be implemented in $O(d)$ time and space, making such 
\emph{linear stochastic approximation (LSA) algorithms} 
an appealing alternative to directly 
computing the solution to $\bar A_t \theta = \bar b_t$, where $\bar A_t = \frac1t\sum_{s=1}^t A_s$, $\bar b_t = \frac1t \sum_{s=1}^t b_s$ by inverting $\bar A_t$ (whose computational cost is greater than $O(d^2)$)

\if0
independent and identically distributed according to $P$ (a distribution with bounded second moment). Here, $P$ denotes the given problem instance. The aim here is to compute a $\ts\in\R^d$ (a minimum or a fixed point) such that $\EE{A_t}\ts=\EE{b_t}$, where $\E$ is the expectation. When $A_t\,$s are rank-$1$ matrices, as is the case in a number of applications to be discussed soon, $b_t -A_t\theta_{t-1}$ can be computed in $O(d)$, which makes them suitable for high dimensional problems. Some examples of such LSA algorithms include the stochastic gradient descent algorithm (SGD) for the problem of linear prediction \cite{bach,bachaistats}, and the \emph{temporal difference} (TD) class of learning algorithms for approximate policy evaluation (APE) in reinforcement learning (RL) \cite{sutton,konda-tsitsiklis,KoTsi03LSA,gtd,gtd2,gtdmp}.
\todoc{konda-tsitsiklis reference resolved to tsitsiklis-van-roy. is this what you want?? I also added KoTsi03:LSA, maybe that's what you wanted.}
\fi
%An additional feature in these class of applications is that $A_t$ turns out to be a rank-$1$ matrix and $A_t\theta_{t-1}$ can be obtained in $O(d)$, which is attractive due to the cheap per time step computational requirement.\par

Assuming sufficient regularity of $\{(A_t,b_t)\}_{t\ge 1}$, e.g., independence, or mixing, in addition to bounded moments, if the stepsize sequence converges to zero at an appropriate rate,
convergence of $\{\theta_t\}_{t\ge 0}$ to $\ts$ can be guaranteed in various senses. \todoc{cite some papers that do this..}
In applications, one often starts from some additional broad properties of the common distribution
$P$ underlying $\{(A_t,b_t)\}$, i.e., $P\in \cP$ for a known family of instances $\cP$.
For example, in \emph{linear regression under the squared loss criterion} (LS), 
$A= \EE{A_t}$ is symmetric and positive definite and $\EE{\normsm{b_t}^2}$, 
$\EE{\normsm{A_t}^2}\le B$ with $B$ known.
The goal then is not only to guarantee asymptotic convergence on a per-instance basis, 
but also to choose $\{\alpha_t\}_{t\ge 1}$ based on the knowledge of $\cP$ only, 
so that the worst-case error is  ``small'' 
over the whole class $\cP$ and for any  $t\ge 1$.

To overcome the difficulty of choosing such a ``universally good'' stepsize sequence,
following the ideas of \citet{ruppert} and \citet{polyak-judisky},
in the context of linear regression with the squared loss (LS),
\citet{bach-moulines} suggested that
\eqref{eq:lsaintro} should be used with $\alpha_t=\alpha>0$ ($t\ge 1$) 
with some $\alpha>0$ to be chosen based on $\cP$, 
and output the average $\thh_t\eqdef\frac{1}{t+1}\sum_{s=0}^t \theta_s$. 
\todoc{I think there is also an impossibility result for choosing $\{\alpha_t\}_{t\ge 1}$,
at least for LS problems. Namely, no universal stepsize sequence achieves the optimal bound.}
The main result of 
\citet{bach-moulines}, 
%which is further improved by  \citet{bach}, 
is that for the LS problem, 
under the assumption that $\{(A_t,b_t)\}_{t\ge 1}$ is an independent sequence, 
the constant stepsize $\alpha$ can be chosen solely based on the above-mentioned 
upper bound $B$ to guarantee that for some universal constant $C>0$
the expected squared prediction error of using $\thh_t$ is at most $C B/t$ for any $t\ge 1$,
which, up to constant factors, is information-theoretically optimal. \todoc{This needs a reference.}

In this paper we ask to what extent the nice result 
of \citet{bach-moulines} can be extended beyond the LS problem;
in other words, we are asking which aspects of the LS problem play a critical role. 
Our interest stems from the desire to repeat this result for
linear value-function estimation (LVE) in reinforcement learning (RL) where
multiple members of the
\emph{temporal-difference (TD) family of algorithms} 
(cf. \cite{sutton,gtd,gtd2,gtdmp} and \cref{sec:rl})
have been proposed as an analog of the ``LMS algorithm'' available for LS.
% that are meant to be an efficient way of estimating value functions
%in reinforcement learning \cite[e.g.,][]{sutton,konda-tsitsiklis,KoTsi03LSA,gtd,gtd2,gtdmp}.
The extension is not straightforward as there are a number of critical differences
between the properties of the instances in LS and LVE.
In particular,  in LVE
{\em (i)}  $A=\EE{A_t}$ is in general non-symmetric;
{\em (ii)} the sequence  $\{(A_t,b_t)\}_{t\ge 1}$ is ``driven by Markov noise'' \cite{tsivan95,tsivan97};
{\em (iii)} and while the natural error metric in LS problems is $\normsm{\thh_t-\ts}_{A}^2$
(here $\norm{x}_A^2 = x^\top\!\!A x$ for $A$ symmetric, positive definite), 
this is not the case in TD.
Of these differences we only consider (i) and (iii), assuming that $\{(A_t,b_t)\}_{t\ge 1}$ is an i.i.d. sequence. For our results that have a negative character these is no loss of generality
making this assumption. 
The question as to what extent our other results can be extended to the Markov noise
case remains for future work. \todoc{I don't have high hopes.}

\iffalse
\if0
In particular, the convergence rates can degrade, or they may depend on potentially unbounded problem dependent constants \cite{bach-moulines}. Diminishing stepsizes such as $\alpha_t=\frac{c_0}{t+c}$, with problem instance specific tuning of the constants $c>0,c_0>0$ have been used in practice \cite{gtd2,gtdmp,konda-tsitsiklis}. 
An alternate idea, which we call the constant stepsize averaged LSA (CALSA) is to run \eqref{eq:lsaintro} by choosing $\alpha_t=\alpha>0$ $\forall t> 0$ with some $\alpha>0$, and output the average $\thh_t\eqdef\frac{1}{t+1}\sum_{s=0}^t \theta_s$. Thus, in CALSA, $\theta_t$ is an internal variable and $\thh_t$ is the output of the algorithm. The idea is that the constant stepsize leads to faster forgetting of initial conditions, while the averaging on the top reduces noise. This idea goes back to  \citet{ruppert} and \citet{polyak-judisky} who proposed it in the context of stochastic approximation that LSA is a special case of.   
\fi

\textbf{Design Questions:} A useful design criterion for CALSA algorithms would be to require them to work in a problem instance independent manner for a given class of  problems. We break down this design criterion into the following two question: $(i)$ whether there exists a \emph{universal} constant stepsize, and $(ii)$ whether the behavior of the algorithm is \emph{uniform} across all the problem instances of the class. The idea is that the universal stepsize can be used across all the instances thereby alleviating the need for an instance dependent tuning of the stepsizes. In this paper, we measure the performance of the algorithm by the mean squared error $\E[\normsm{\thh_t-\ts}^2_M]$, where $M$ is a positive definite matrix (whose choice plays a critical role). Here, we look at algorithm behavior in the finite-time, as well as asymptotic (a weaker condition) sense. By uniform finite-time behavior, we mean a convergence rate of $\frac{C}{t}$ for the MSE, where the constant $C>0$ is independent of the problem instance. And asymptotic fast convergence rate of $O(\frac{1}{t})$ means the constant in the rate expression can be problem instance dependent.

\textbf{Motivation:} Recently, \citet{bach} considered, what we call a constant stepsize averaged stochastic gradient descent (CASGD) 
 algorithm for the linear least squares prediction problem (with \iid sampling) and showed that there exists both universal stepsize and a uniform convergence rate of $\frac{C}{t}$ for the mean squared prediction error.

\textbf{Focus:} We are interested in the TD class of algorithms that are also LSA algorithms. Specifically, we look at CATD(0) and CAGTD algorithms, which are TD(0) and GTD algorithms with constant stepsize and iterate averaging. Here, we want to repeat the feat of \cite{bach}, i.e., we want to find out whether the CATD(0) and CAGTD achieve a uniform convergence rate with a universal stepsize for useful classes of approximate policy evaluation problems. 
\fi
We answer these question in the following manner:
%%%%%%%%%%%%%%%%%%%%%%%%%%%%%%%%%%%%%%%%%%
\begin{enumerate}[topsep=0pt,itemsep=1pt,wide, labelwidth=!, labelindent=0pt,label=\emph{\arabic*}.]
%%%%%%%%%%%%%%%%%%%%%%%%%%%%%%%%%%%%%%%%%%
\item \textbf{Finite-time Instance Dependent Bounds} (\Cref{sec:mainresults}): 
When $\EE{A_t}$ is Hurwitz, we show that under additional mild regularity assumptions 
there exists a constant $\alpha_P>0$ such that for any $\alpha\in (0,\alpha_P)$,
the mean-squared error (MSE), $\EE{\normsm{\thh_t-\ts}^2}$
is at most $\frac{C}{t}+\frac{C'}{t^2}$ 
with some positive constants $C,C'$ that we explicitly compute from $P$ and $\alpha$
(\cref{th:rate}).
Our result is an extension of the result by \citet{polyak-judisky} who proved a similar result
for $A_t=A$. We also show that our upper bound is essentially tight up to a universal constant
factor (\cref{th:lb}).
%%%%%%%%%%%%%%%%%%%%%%%%%%%%%%%%%%%%%%%%%%
\item \textbf{Problem Landscape} (\Cref{sec:land}): 
By means of a simple example we establish that 
Hurwitzness and uniformly bounded moments
alone are insufficient for the existence of a single \emph{universal} stepsize (\cref{prop:unistep}).
Here, a universal stepsize would guarantee that the worst-case expected squared 
error over the chosen class is controlled, an in particular converges to zero.
We also present a number of simple problem classes, closely related to the LS problem,
that are used to highlight the significance of various properties of LS problems,
such as the positive definiteness of $A$, that the error is measured in norm $\norm{\cdot}_A$,
and the so-called ``structured noise'' property (\cref{th:pspd}).
The strength of our results are that they give the exact behavior of the worst-case error (i.e.,
matching lower and upper bounds).
%%%%%%%%%%%%%%%%%%%%%%%%%%%%%%%%%%%%%%%%%%
\item \textbf{Reinforcement Learning} (\Cref{sec:rl}): 
In the context of reinforcement learning we establish that the constant stepsize averaged TD($0$) 
and a novel version of GTD assume universal stepsizes in a number of cases 
for various problem classes with bounded data (\cref{th:tdadmis}).
In particular, the result is shown for averaged TD($0$) for the ``on-policy case'', which we define
slightly differently from its standard formulation, which is necessary given that we consider the i.i.d. case. We also show that a universal stepsize exist for averaged TD($0$) 
even for the general (non-on-policy case) provided that 
the features are normalized and the data is ``stationary''. 
Finally, we establish that even the normalization condition can be dropped for the novel version of GTD. The strength of these results is that they free the user 
concerned with achieving the $O(1/t)$ problem-dependent rate from the burden of 
designing stepsize tuning methods.
In \cref{sec:exp} we illustrate these theoretical results by means of simulated computer
experiments.
%%%%%%%%%%%%%%%%%%%%%%%%%%%%%%%%%%%%%%%%%%
%%%%%%%%%%%%%%%%%%%%%%%%%%%%%%%%%%%%%%%%%%
\end{enumerate}%In the main body of the paper we present the results and the proofs are presented in the supplementary material. 
%%%%%%%%%%%%%%%%%%%%%%%%%%%%%%%%%%%%%%%%%%
After our results, we briefly discuss related work in \cref{sec:related}.
In connection to this, we also wish to mention
that other computationally cheap methods such as those based on matrix sketching idea, could be a viable alternative to  CALSA. \todoc{Add some ref! Book..? Widgerson?}
However, especially in RL, LSA has been at the heart of the widely used TD algorithms for more than a decade, and our work contributes to the understanding of the limitations and benefits of such algorithms. Also, we believe understanding CALSA is foundational in the sense that in most cases of general incremental algorithms our understanding is based on their ``linearized'' versions. We now present some basic notations followed throughout the paper.

